%%%%%%%%%%%%%%%%%%%%%%%%%%%%%%%%%%%%%%%%%
% Developer CV
% LaTeX Template
% Version 1.0 (28/1/19)
%
% This template originates from:
% http://www.LaTeXTemplates.com
%
% Authors:
% Jan Vorisek (jan@vorisek.me)
% Based on a template by Jan Küster (info@jankuester.com)
% Modified for LaTeX Templates by Vel (vel@LaTeXTemplates.com)
%
% License:
% The MIT License (see included LICENSE file)
%
%%%%%%%%%%%%%%%%%%%%%%%%%%%%%%%%%%%%%%%%%

%----------------------------------------------------------------------------------------
%	PACKAGES AND OTHER DOCUMENT CONFIGURATIONS
%----------------------------------------------------------------------------------------

\documentclass[10pt]{tpl/developercv} % Default font size, values from 8-12pt are recommended
\usepackage[russian]{babel}
\usepackage[hidelinks,
            colorlinks = true,
            linkcolor = blue,
            urlcolor  = blue,
            citecolor = blue,
            anchorcolor = blue]{hyperref}
\usepackage{datetime}
\usepackage{texdate}
\usepackage{enumitem}
\initcurrdate
\def\setdateformat{d-m-Y}
%----------------------------------------------------------------------------------------


%----------------------------------------------------------------------------------------
% Header & Footer
%----------------------------------------------------------------------------------------
\usepackage{fancyhdr}

\pagestyle{fancy}
\fancyhf{}
\rhead{\href{https://rame0.github.io/resume-latex/resume_ru.pdf}{Скачать актуальную версию}}
\lhead{Док. обновлен: \printdate\ \currenttime}
% \rfoot{Page \thepage}
%----------------------------------------------------------------------------------------

\begin{document}

%----------------------------------------------------------------------------------------
%	TITLE AND CONTACT INFORMATION
%----------------------------------------------------------------------------------------

\begin{minipage}[t]{0.44\textwidth} % 45% of the page width for name
	\vspace{-\baselineskip} % Required for vertically aligning minipages

	% If your name is very short, use just one of the lines below
	% If your name is very long, reduce the font size or make the minipage wider and reduce the others proportionately
	\colorbox{black}{{\Huge\textcolor{white}{\textbf{\MakeUppercase{Рамиль}}}}} % First name

	\colorbox{black}{{\Huge\textcolor{white}{\textbf{\MakeUppercase{Алиякберов}}}}} % Last name

	\vspace{6pt}

  \bigskip
	{\Large Full stack web-разработчик} % Career or current job title

  \vspace{6pt}

  \textbf{\underline{RU}} / \href{https://rame0.github.io/resume-latex/resume_en.pdf}{EN}

\end{minipage}
\begin{minipage}[t]{0.275\textwidth} % 27.5% of the page width for the first row of icons
	\vspace{-\baselineskip} % Required for vertically aligning minipages

	% The first parameter is the FontAwesome icon name, the second is the box size and the third is the text
	% Other icons can be found by referring to fontawesome.pdf (supplied with the template) and using the word after \fa in the command for the icon you want
	\icon{MapMarker}{12}{\ Москва, Россия}

	\href{mailto:r@me0.biz}{\icon{At}{12}\ r@me0.biz}

	\href{tel:+79253943150}{\icon{Phone}{12}\ +7 925 394-31-50}
\end{minipage}
\begin{minipage}[t]{0.275\textwidth} % 27.5% of the page width for the second row of icons
	\vspace{-\baselineskip} % Required for vertically aligning minipages

	% The first parameter is the FontAwesome icon name, the second is the box size and the third is the text
	% Other icons can be found by referring to fontawesome.pdf (supplied with the template) and using the word after \fa in the command for the icon you want
	\href{https://www.rame0.ru/}{\icon{Globe}{12}\ www.rame0.ru}

	\href{https://github.com/rame0}{\icon{Github}{12}\ github.com/rame0}

	\href{https://stackoverflow.com/users/513723/rame0}{\icon{StackOverflow}{12}\ StackOverflow}

\end{minipage}

\vspace{0.5cm}

%----------------------------------------------------------------------------------------
%	INTRODUCTION, SKILLS AND TECHNOLOGIES
%----------------------------------------------------------------------------------------

\cvsect{Кто я?}


\begin{minipage}[t]{0.45\textwidth} % 40% of the page width for the introduction text
	\vspace{-\baselineskip} % Required for vertically aligning minipages

	Разработчик с стажем более 10 лет.
  \bigskip
  \small{
    \begin{itemize}[leftmargin=0pt,rightmargin=0]
      \item На последних курсах университета работал C\# разработчиком.
      \item После службы в ВС РФ работал "'отделом разработки"' в небольшой семейной компании (PHP, JS, CSS).
      \item После этого некоторое время занимался SEO (продвижение сайтов в поисковых системах) и преподавал его, совмещая с разработкой сайтов на фрилансе.
      \item На данный момент работаю аутсорс-разработчиком в оптовой компании (PHP, JS, SCSS).
      \item Основные используемые технологии: PHP, JS (в основном jQuery), SCSS, Docker, Yarn, Composer, Gulp, {Git (GitHub)}.
      \item Периодически используемые: Python, редко C\#.
      \item В подчинении 1 программист. Управление задачами YouTrack (Agile)
    \end{itemize}
  }
\end{minipage}
\hfill % Whitespace between
\begin{minipage}[t]{0.5\textwidth} % 50% of the page for the skills bar chart
	\vspace{-\baselineskip} % Required for vertically aligning minipages
	\begin{barchart}{5.5}
		\baritem{JavaScript}{90}
		\baritem{PHP}{100}
		\baritem{MYSQL}{100}
		\baritem{SASS/LESS}{70}
		\baritem{Git}{60}
		\baritem{Docker(compose)}{50}
		\baritem{NPM}{80}
		\baritem{Composer}{80}
	\end{barchart}
\end{minipage}

% \begin{center}
% 	\bubbles{5/PHPStorm, 4/Linux DE, 3/Linux Server, 5/Windows}
% \end{center}

%----------------------------------------------------------------------------------------
%	EXPERIENCE
%----------------------------------------------------------------------------------------

\cvsect{Опыт}

\begin{entrylist}
	\entry
		{2007 -- 2008

		\footnotesize{part time}}
		{Junior C\# Developer}
		{ООО 'Группа Алланд'}
		{Разработка пользовательских интерфейсов, архитектуры системы и архитектуры баз данных(в меньшей степени). Мы разрабатывали информационные системы, на подобии 1С, но со специфическими задачами. Так же нами разрабатывалась система контроля полного цикла проведения всероссийских интеллектуальных конкурсов для школьников.

		\texttt{C\#}\slashsep\texttt{SQL}\slashsep\texttt{Web/Win-Services}\slashsep\texttt{OLAP}\slashsep\texttt{TFS}}
	\entry
		{2008-2009}
		{Срочная служба}
		{ВС РФ}
		{Ничего особенного, иногда помогал с презентациями, офисными программами и программами учета.}
	\entry
		{2009 -- 2015}
		{Full stack разработчик}
		{www.evrosite.ru}
		{Работа в небольшой семейной компании. Eдинственный разработчик. Время от времени, к работе привлекались фрилансеры, которым я раздавал задачи и проверял качество исполнения.

		Выполнял весь цикл разработки сайтов, начиная с верстки и заканчивая размещением на хостинге/сервере (включая поддержку серверов). В основном, разработка велась на внутренней CMS разработанной предыдущим программистом. Так же, ипользовались такие CMS, как: OpenCart, 1C Битрикс, AmiroCMS, Wordpress. В сложные времена, так же, выполнял работу продажника, менеджера проектов и другие обязанности.

		\texttt{PHP}\slashsep\texttt{JS}\slashsep\texttt{CSS}\slashsep\texttt{MySQL}\slashsep\texttt{Linux}}

	\entry
		{4/2013 -- 2/2018

		\footnotesize{part time}}
		{SEO специалист/Full stack}
		{Фриланс}
		{С конца 2013 года, когда кол-во заказов стало уменьшаться, заинтересовался SEO продвижением и, понемногу, начала оказывать услуги SEO.\\
		С 2015 основную часть заказов составляли заказы по SEO. С уклоном в фишки, которые можно внедрять с позиции разработчика. Так сказать, базовое продвижение, о котором большинство разработчиков сайтов даже не задумывалось.\\
		По прежнему занимался разработкой сайтов но уже в меньшем количестве (до 10 сайтов в год). В основном делала сайты на AmiroCMS и, иногда, на 1С Битрикс

		\texttt{PHP}\slashsep\texttt{JS}\slashsep\texttt{CSS}\slashsep\texttt{MySQL}\slashsep\texttt{Linux}}
	\entry
		{3/2015 -- 4/2017

		\footnotesize{part time}}
		{Преподаватель SEO}
		{ООО "'ТопЭксперт"' (www.topexpert.pro)}
		{Переподавал базовые знания по Поисковой оптимизации -- SEO (Search Engine Optimisation) -- на образовательной платформе TopЭксперт. Провел в общей сложности 7 потоков, более 100 человек. В обязанности входило проведение уроков теория/практика и проверка домашних заданий. Основной упор в базовых знаниях делался на понимании что такое поисковая система и как она работает, а так же на том, какие ошибки (с точки зрения SEO) допускают разработчики сайтов и как их исправить. Принимал участие в разработке плана обучения, самого курса, а так же доп. материалов(презентаций, примеров и прочего). Написал руководство по составлению семмантического ядра сайта.}
	\entry
		{3/2016 -- 2/2018

		\footnotesize{part time}}
		{Full stack outsource}
		{ООО "'Интерьер комфорт"' }
		{Помощь в поддержке серверов. Доработка и поддержка 2 интернет-магазинов. Поддерка системы управления складом и оптовыми продажами.

		\texttt{PHP}\slashsep\texttt{JS}\slashsep\texttt{SCSS}\slashsep\texttt{MySQL}\slashsep\texttt{Linux}}
	\entry
		{2/2018 -- present}
		{Full stack outsource}
		{ООО "'Интерьер комфорт"' }
		{
    Перешел в компанию на full-time.

    Разработка с нуля системы складов и оптовых продаж. Разработка внутренней CMS для данной системы.

		Перевод интернет магазина на внутреннюю CMS.

		Выделение ядна внутренней CMS в отдельный проект.

		Разработка системы для управления перевозками (транспортной компании) на базе разработанного ядра.

		С начала 2021 года 1 программист в подчинении.

		\texttt{PHP}\slashsep\texttt{JS}\slashsep\texttt{SCSS}\slashsep\texttt{MySQL}\slashsep\texttt{Linux}}
\end{entrylist}

%----------------------------------------------------------------------------------------
%	EDUCATION
%----------------------------------------------------------------------------------------

\cvsect{Образование}

\begin{entrylist}
	\entry
		{2003 -- 2008}
		{Мат. обеспечение и админ. информационных систем}
		{УлГУ}
		{Получил диплом о высшем образовании по специальности Математическое обеспечение и администрирование информационных систем в Ульяновском Государственном Университете.

		Тема диплома: Разработка автоматизированной системы технической поддержки пользователей, на основе встраиваемого интернет пейджера (.NET).

		Технологии: C\#, MSSQL, Client-Server
		}
	\entry
		{constant}
		{Самообразование}
		{}
		{Стараюсь изучать новые технологии и улучшать знания в имеющихся. К сожалению, не всегда получается изучить то, что на данный момент не применимо. Например, давно хочется изучить ReactJS, но дальше чем пройти пару туториалов дело не заходило, т.к. пока это не применимо к рабочим проектам.}
\end{entrylist}

%----------------------------------------------------------------------------------------
%	ADDITIONAL INFORMATION
%----------------------------------------------------------------------------------------

\begin{minipage}[t]{0.3\textwidth}
	\vspace{-\baselineskip} % Required for vertically aligning minipages

	\cvsect{Languages}

	\textbf{Русский} - родной\\
	\textbf{English} - говорю/пишу\\
\end{minipage}
\hfill
\begin{minipage}[t]{0.3\textwidth}
	\vspace{-\baselineskip} % Required for vertically aligning minipages

	\cvsect{Хобби}

	Часто смотрю что-то образовательное. Изучаю разные штуки, в основном связанные с компьютерами. Например Pen-тестинг или 3d моделирование.
	Реже - читаю sci-fi и фентези.
\end{minipage}
\hfill
\begin{minipage}[t]{0.3\textwidth}
	\vspace{-\baselineskip} % Required for vertically aligning minipages

	\cvsect{Non profit}

	Изредки пишу что то в блог и отвечаю на StackOverflow. Иногда занимаюсь ревью постов на StackOverflow.
\end{minipage}

%----------------------------------------------------------------------------------------

%----------------------------------------------------------------------------------------
% MORE ABOUT PROJECTS
%----------------------------------------------------------------------------------------
\newpage
\cvsect{\HugeПроекты}

На подавляющем большинстве проектов использовались: PHP (от 5.1 до 7.4), JQuery, MySQL.

\section{Разработка более 200 сайтов}

За все время работы в веб-студии и во время работы в SEO, мной и с моим участием, было создано более 200 вебсайтов разного типа. От визиток до интернет-магазинов. Изредка попадались интересные проекты вроде:
\begin{itemize}
   \item разработки тестированя сотрудников (PHP),
   \item разработки софта для опроса и последующей разработки рекомендаций для бизнеса (C\#),
   \item разработки одностраничника с большим количеством меняющейся анимации при скроле (JavaScript)
   \item разработка системы для управления складами и оптовыми продажами (PHP, JS)
\end{itemize}

В подавляющем большинстве случаев, сайты разрабатывались на внутренней CMS, созданной для простых сайтов еще в 2005 на PHP 5.1. Она использовалась вплоть до 2015 года. Попытки ее доработки не прекращалсь до полного отказа. Она была не гибкая, но очень простая и быстрая в разработке, поэтому её оставляли для чего-то что нужно сделать "вчера". По сути, вся CMS была одним файлом (если не считать файлы с логикой авторизации, работы с БД, и рендера самого сайта) с большим количеством *if'ов* и кусков *HTML*. На ней же разработанна система управления складами и продажами для поставщика ковров, на которую я частично перешел работать в 2016 году.

\begin{itemize}
 \item Большим плюсом и, в тоже время минусом, системы было то, что всё (кроме главного файла) можно было изменять непосредственно в самой системе, включая таблицы в БД.

 \item Минусом же были сложность добавления функционала, невозможность переиспользовать код (которая частично исправлялась со временем), а так же "дырявость" и устаревшая версия языка.
\end{itemize}

\section{Система управленя складом и оптовыми продажами}
\subsection{Версия 1}

Первая версия системы была написана в 2013 году. Для её разработки была использованна упомянутая ранее CMS с огромным количеством доработок. Некоторые из них:
\begin{itemize}
  \item Была переписана система авторизации.
  \item Переписана сама БД и работа с ней.
  \item Переписан, частично, "шаблонизатор".
  \item Были написаны несколько видов API: для AJAX запросов, для партнеров, для маркетплейсов.
\end{itemize}

Самыми большими недостатками явились:
\begin{itemize}
  \item Минимальная связанность частй системы.
  \item Невозможность добавления складов, в результате чего, для разных складов приходилось создавать отдельные копии системы.
  \item Устаревший код который стало, практически, невозможно поддерживать и дополнять.
\end{itemize}

В таком виде система просуществовала до 2018 года пока небыло принято решение её написать ее с нуля.

\subsection{Версия 2}
В 2018 году удалось убедить руководство, что дальнейшая поддержка системы практически невозможна. И было принято решение начать разрабатывать новую версию с нуля.

Она уже разрабатывалась на шаблону MVC*. Стандартные логические слои: Данные - Логика - Представления.
\paragraph{}

Backend:
\begin{itemize}
  \item Роутер собственного производства, похожий на FastRoute
  \item За работу с БД отвечает FluentPDO
  \item Шаблонизатор - Twig
  \item Кеширование шаблонов Twig делает сам, кеширование других данных реализованно на файлах
  \item Для работы Моделей, реализован класс Model наследуемый от Entity (общее описание структуры) и реализующий методы выбора/изменения/сохранения этой структуры запросами к БД через FluentPDO
  \item Для работы с наборами данных реализованны классы Collection и Dictionary (используется редко)
  \item Создание основных объектов, вроде App или Request, реализованно через шаблон Singleton
  \item Для отладки используется мой форк умершего \href{https://github.com/rame0/chromephp}{ChromePHP} (чуть более старая версия, т.к. новому логеру нужен PHP7.4)
  \item PHP 7.2 + MySQL (MariaDB)
\end{itemize}

Frontend:
\begin{itemize}
  \item JQuery - JS фреймворк
  \item Bootstrap - CSS фремворк
  \item Все CSS стили пишутся на SASS
  \item Сборка и минификация JS и SCSS осуществляется с помощью Gulp
\end{itemize}

Other:
\begin{itemize}
  \item Docker используется для разворачивания DEV среды
  \item Composer - менеджеры пакетов PHP
  \item Yarn - менеджеры пакетов JS
  \item Реализованы Yarn скрипты для:
  \begin{itemize}
    \item запуска build (dev/prod) и watch для сборки JS и CSS
    \item установки и обновления пакетов Composer в Docker образе
    \item запуска выполнения Cron задач в образе
  \end{itemize}
\end{itemize}

\subsection{Основные функции}
\begin{itemize}
  \item Управление клиентами
  \item Управление остатками товаров
  \item {
    Удобный табличный вид оформления заказа с навигацией стралками клавиатуры
    \begin{center}
      \begin{tabular}{ |c|c|c| }
        \hline
        & 2x3 & 3x4 \\
        \hline
         Ковёр 1 & [  ] & [  ] \\
                & 1 шт & 5 шт \\
        \hline
         Ковёр 2 & [] & [  ] \\
                & 0 шт & 10 шт \\
        \hline
      \end{tabular}
    \end{center}
  }
  \item Управление коллекциями, товарами и ценами
  \item Рассылки клиентам различных уведомлений
  \item Экспорт/импорт товаров и остатков
  \item API для партнеров (восстановление из v1 в процессе)
  \item API для маркетплейсов (только в v1)
\end{itemize}

\section{Основной розничный сайт}
В 2020 году на базу разработанного кода системы управления был перенесен основной розничный сайт компании (\href{https://www.kovrotexs.ru/}{https://www.kovrotexs.ru/}).

На первой версии сайта Back'ом и Front'ом занимался только я. Во время перехода на новый backend, для ускорения работы, для реализации front'а был нанят фрилансер (я только сделал ему базу и следил за выполнением).

\begin{itemize}
  \item Frontend, по прежнему, JQuery + Bootstrap
  \item Были добавлены методы кеширования каталогов, фильтров и характеристик товаров.
  \item В качестве поискового движка используется Manticore Search.
  \item Полностью переработаны методы работы с товарами и остатками.
  \item Управление повторяющимися задачами - Crunz (Cron)
  \item Реализован клиент API для получения остатков с сайтов партнеров.
\end{itemize}

\section{Система управления транспортной компанией}
Во втором квартале 2021 года началась работа над созданием системы для управления перевозками. Велись обсуждения и исследования и прикидывались основная схема работы и общий функционал.

В конце лета 2021 началась основная фаза разработки.

\begin{itemize}
  \item Frontend - JQuery + Bootstrap
  \item PHP 7.4
  \item Было принято решение выделить предыдущие наработки в отдельный фрейворк и работать на нем
  \item В процессе переноса отказался от Singleton'ов
  \item Заменил собственный роутер на FastRoute
  \item Сборщик Glup заменен на Webpack (он показал себя минимум в 10 раз быстрее)
  \item На данный момент разработка еще ведется. В целом система готова (прием грузов и заказов, погрузка/разгрузка, управление сотрудниками/клиентами/водителями/машинами). Идет тестирование и, часто, полное изменение некоторого функционала.
\end{itemize}

%----------------------------------------------------------------------------------------

\end{document}
