
%----------------------------------------------------------------------------------------
%	PACKAGES AND OTHER DOCUMENT CONFIGURATIONS
%----------------------------------------------------------------------------------------

\documentclass[10pt]{tpl/developercv} % Default font size, values from 8-12pt are recommended
\usepackage[russian]{babel}
\usepackage[hidelinks,
            colorlinks = true,
            linkcolor = blue,
            urlcolor  = blue,
            citecolor = blue,
            anchorcolor = blue]{hyperref}
\usepackage{datetime}
\usepackage{texdate}
\usepackage{enumitem}
\initcurrdate
\def\setdateformat{d-m-Y}
%----------------------------------------------------------------------------------------


%----------------------------------------------------------------------------------------
% Header & Footer
%----------------------------------------------------------------------------------------
\usepackage{fancyhdr}

\pagestyle{fancy}
\fancyhf{}
\lhead{Док. обновлен: \printdate\ \currenttime}
\rhead{\href{https://rame0.github.io/resume-latex/resume_ru.pdf}{Скачать актуальную версию}}
\lfoot{Создано на \LaTeX}
\rfoot{\href{https://github.com/rame0/resume-latex}{Исходный код резюме}}
%----------------------------------------------------------------------------------------



\begin{document}


%----------------------------------------------------------------------------------------
%	TITLE AND CONTACT INFORMATION
%----------------------------------------------------------------------------------------

\begin{minipage}[t]{0.44\textwidth} % 45% of the page width for name
	\vspace{-\baselineskip} % Required for vertically aligning minipages

	% If your name is very short, use just one of the lines below
	% If your name is very long, reduce the font size or make the minipage wider and reduce the others proportionately
	{\Huge{\textbf{\textcolor{mainblue}{\MakeUppercase{Рамиль}}}}} % First name

	\vspace{4pt}

	{\Huge{\textbf{\textcolor{mainblue}{\MakeUppercase{Алиякберов}}}}} % Last name

	\vspace{6pt}

  \large{Full stack web-разработчик} % Career or current job title

\end{minipage}
\begin{minipage}[t]{0.275\textwidth} % 27.5% of the page width for the first row of icons
	\vspace{-\baselineskip} % Required for vertically aligning minipages

	% The first parameter is the FontAwesome icon name, the second is the box size and the third is the text
	% Other icons can be found by referring to fontawesome.pdf (supplied with the template) and using the word after \fa in the command for the icon you want
	\icon{MapMarker}{12}{\ Москва, Россия}

	\href{mailto:r@me0.biz}{\icon{At}{12}\ r@me0.biz}

	\href{tel:+79253943150}{\icon{Phone}{12}\ +7 925 394-31-50}
\end{minipage}
\begin{minipage}[t]{0.275\textwidth} % 27.5% of the page width for the second row of icons
	\vspace{-\baselineskip} % Required for vertically aligning minipages

	% The first parameter is the FontAwesome icon name, the second is the box size and the third is the text
	% Other icons can be found by referring to fontawesome.pdf (supplied with the template) and using the word after \fa in the command for the icon you want
	\href{https://www.rame0.ru/}{\icon{Globe}{12}\ www.rame0.ru}

	\href{https://github.com/rame0}{\icon{Github}{12}\ github.com/rame0}

	\href{https://stackoverflow.com/users/513723/rame0}{\icon{StackOverflow}{12}\ StackOverflow}

\end{minipage}

\vspace{0.5cm}


\vspace{6pt}

\begin{minipage}[t]{0.25\textwidth}
	\vspace{-\baselineskip} % Required for vertically aligning minipages
Language: \textbf{\underline{RU}} / \href{https://rame0.github.io/resume-latex/resume_en.pdf}{EN}
\end{minipage}
\hfill % Whitespace between
\begin{minipage}[t]{0.70\textwidth}
	 \vspace{-\baselineskip} % Required for vertically aligning minipages
Проекты: \href{https://rame0.github.io/resume-latex/projects_ru.pdf}{RU} / \href{https://rame0.github.io/resume-latex/projects_en.pdf}{EN}
\end{minipage}
%----------------------------------------------------------------------------------------
%	INTRODUCTION, SKILLS AND TECHNOLOGIES
%----------------------------------------------------------------------------------------


\begin{minipage}[t]{0.55\textwidth} % 40% of the page width for the introduction text
	\vspace{-\baselineskip} % Required for vertically aligning minipages

  \cvsect{Вкратце обо мне}

	Разработчик со стажем более 10 лет.
  \bigskip
  \small{
    \begin{itemize}[leftmargin=1pt]
      \item На последних курсах университета работал C\# разработчиком.
      \item После службы в ВС РФ работал "'отделом разработки"' в небольшой семейной компании (PHP, JS, CSS).
      \item После этого некоторое время занимался SEO (продвижение сайтов в поисковых системах) и преподавал его, совмещая с разработкой сайтов на фрилансе.
      \item На данный момент работаю аутсорс-разработчиком в оптовой компании (PHP, JS, SCSS).
      \item{Основные используемые технологии: PHP, SCSS, JS (в основном jQuery), Docker, Yarn, Composer, Gulp, Git (GitHub), Webpack (внедрил недавно).}
      \item Периодически используемые: Python, редко C\#.
      \item В разное время брал фрилансеров на подряд.
      \item Сейчас в подчинении 1 программист.
    \end{itemize}
  }
\end{minipage}
\hfill % Whitespace between
\begin{minipage}[t]{0.35\textwidth} % 50% of the page for the skills bar chart
	 \vspace{-\baselineskip} % Required for vertically aligning minipages

   \cvsect{Skills}
% Skill bar section, each skill must have a value between 0 an 6 (float)
\skills{{YouTrack (Agile)/3},{Composer/4},{NPM/4},{Docker/3},{Git/3},{SASS/4},{MySQL/5},{JavaScript/5.2},{PHP/5.5}}
% \skills
%
% \begin{barchart}{5.5}
% 	\baritem{JavaScript}{90}
% 	\baritem{PHP}{100}
% 	\baritem{MySQL}{100}
% 	\baritem{SASS/LESS}{70}
% 	\baritem{Git}{60}
% 	\baritem{Docker(compose)}{50}
% 	\baritem{NPM}{80}
% 	\baritem{Composer}{80}
%   \baritem{YouTrack (Agile)}{50}
% \end{barchart}

\end{minipage}

% \begin{center}
% 	\bubbles{5/PHPStorm, 4/Linux DE, 3/Linux Server, 5/Windows}
% \end{center}

%----------------------------------------------------------------------------------------
%	EXPERIENCE
%----------------------------------------------------------------------------------------

\cvsect{Опыт}

\begin{entrylist}
	\entry
		{2/2018 -- present}
		{Full-stack outsource}
		{ООО "'Интерьер комфорт"' }
		{Перешел в компанию на full-time. С начала 2021 года 1 программист в подчинении.

    Что я делал:
    \small\begin{itemize}
      \item Разработка с нуля системы складов и оптовых продаж. Разработка внутренней CMS для данной системы. \underline{Архитектура, frontend, backend, БД.}
      \item Перевод интернет магазина на внутреннюю CMS.
      \item Выделение ядра внутренней CMS в отдельный проект.
      \item Разработка системы для управления перевозками (транспортной компании) на базе разработанного ядра.
    \end{itemize}


		\texttt{PHP}\slashsep\texttt{JS}\slashsep\texttt{SCSS}\slashsep\texttt{MySQL}\slashsep\texttt{Linux}
    \slashsep\texttt{Docker}\slashsep\texttt{Yarn/NPM}\slashsep\texttt{Composer}\slashsep\texttt{GitHub}

    \texttt{Gulp}\slashsep\texttt{Webpack}}

	\entry
		{3/2016 -- 2/2018

		\footnotesize{part-time}}
		{Full-stack outsource}
		{ООО "'Интерьер комфорт"' }
		{Работал один над всем спектром технических задач компании, в частности:
    \small\begin{itemize}
      \item Помощь в поддержке серверов.
      \item Доработка и поддержка 2 интернет-магазинов.
      \item Поддержка системы управления складом и оптовыми продажами, которую я разработал ранее как фрилансер.
      \item Добавление нового функционала.
    \end{itemize}

		\texttt{PHP}\slashsep\texttt{JS}\slashsep\texttt{SCSS}\slashsep\texttt{MySQL}\slashsep\texttt{Linux}
    \slashsep\texttt{Yarn/NPM}\slashsep\texttt{Composer}\slashsep\texttt{GitHub}\slashsep\texttt{Gulp}
    }

	\entry
		{3/2015 -- 4/2017

		\footnotesize{part-time}}
		{Преподаватель SEO}
		{ООО "'ТопЭксперт"' (www.topexpert.pro)}
		{Преподавал базовые знания по Поисковой оптимизации (SEO) на образовательной платформе TopЭксперт. Провел в общей сложности 7 потоков студентов, более 100 человек. Основной упор в базовых знаниях делался на понимании, что такое поисковая система и как она работает, а так же, какие ошибки (с точки зрения SEO) допускают разработчики и как их исправить. Написал руководство по составлению семантического ядра сайта.

    В обязанности входило:
    \small\begin{itemize}
      \item проведение уроков теория/практика
      \item проверка домашних заданий
      \item принимал участие в разработке плана обучения и самого курса
      \item создание доп. материалов(презентаций, примеров и прочего).
    \end{itemize}
    }

	\entry
		{4/2013 -- 2/2018

		\footnotesize{part-time}}
		{SEO специалист/Full-stack}
		{Фриланс}
		{С конца 2013 года, заинтересовался SEO продвижением и, понемногу, начала оказывать услуги SEO.
		С 2015 основную часть заказов составляли заказы по SEO. С уклоном в фишки, которые можно внедрять с позиции разработчика. Те, о которых на этапе разработки ни кто не задумывается.\\
		Продолжал разрабатывать сайты, но уже в меньшем количестве (до 10 сайтов в год). В основном делала сайты на AmiroCMS и, иногда, на 1С Битрикс

		\texttt{SEO}\slashsep\texttt{PHP}\slashsep\texttt{JS}\slashsep\texttt{CSS}\slashsep\texttt{MySQL}\slashsep\texttt{Linux}}

  \entry
    {2009 -- 2015}
    {Full-stack разработчик}
    {www.evrosite.ru}
    {Работа в небольшой семейной компании. Единственный разработчик. Время от времени, привлекали фрилансеров, которым я ставил задачи и следил за качеством.\\
    Выполнял весь цикл разработки сайтов, от верстки до размещением на хостинге (включая поддержку серверов). В основном, разработка велась на внутренней CMS. Так же использовали CMS: OpenCart, 1C Битрикс, AmiroCMS, Wordpress. В сложные времена, так же, выполнял работу продажника, менеджера проектов и другие обязанности.

    \texttt{PHP}\slashsep\texttt{JS}\slashsep\texttt{CSS}\slashsep\texttt{MySQL}\slashsep\texttt{Linux}}

	\entry
		{2007 -- 2008

		\footnotesize{part-time}}
		{Junior C\# Developer}
		{ООО 'Группа Алланд'}
		{Мы разрабатывали информационные системы, на подобии 1С, но со специфическими задачами. Так же нами разрабатывалась система контроля полного цикла проведения всероссийских интеллектуальных конкурсов для школьников.

    Мои обязанности:
    \begin{itemize}
      \item разработка пользовательских интерфейсов
      \item тестирование пользовательских интерфейсов
      \item разработка простых контроллеров и моделей
      \item архитектуры системы и архитектуры баз данных(в меньшей степени)
    \end{itemize}

		\texttt{C\#}\slashsep\texttt{SQL}\slashsep\texttt{Web/Win-Services}\slashsep\texttt{OLAP}\slashsep\texttt{TFS}}

\end{entrylist}

%----------------------------------------------------------------------------------------
%	EDUCATION
%----------------------------------------------------------------------------------------

\cvsect{Образование}

\begin{entrylist}
	\entry
		{2003 -- 2008\\\footnotesize{Бакалавриат\\очно}}
		{Мат. обеспечение и админ. информационных систем}
		{УлГУ}
		{Получил диплом о высшем образовании по специальности Математическое обеспечение и администрирование информационных систем в Ульяновском Государственном Университете.

		Тема диплома: Разработка автоматизированной системы технической поддержки пользователей, на основе встраиваемого интернет пейджера (.NET).

		Технологии: C\#, MSSQL, Client-Server
		}
	\entry
		{constant}
		{Самообразование}
		{}
		{Стараюсь изучать новые технологии и улучшать знания в имеющихся. Периодически пробую машинное обучение. Проходил туториалы по ReactJS и Go. Слушал лекции по функциональному программированию и Haskel в частности. К сожалению, сложно изучать то, что на данный момент не применимо, так как негде применить. }
\end{entrylist}

%----------------------------------------------------------------------------------------
%	ADDITIONAL INFORMATION
%----------------------------------------------------------------------------------------

\begin{minipage}[t]{0.3\textwidth}
	\vspace{-\baselineskip} % Required for vertically aligning minipages

	\cvsect{Languages}

	\textbf{Русский} - родной\\
	\textbf{English} - говорю/пишу\\
\end{minipage}
\hfill
\begin{minipage}[t]{0.3\textwidth}
	\vspace{-\baselineskip} % Required for vertically aligning minipages

	\cvsect{Хобби}

	Часто смотрю что-то образовательное. Изучаю разные штуки, в основном связанные с компьютерами. Например, Pen-тестинг или 3d моделирование.
	Реже - читаю sci-fi и фентези.
\end{minipage}
\hfill
\begin{minipage}[t]{0.3\textwidth}
	\vspace{-\baselineskip} % Required for vertically aligning minipages

	\cvsect{Non profit}

	Изредка пишу что-то в блог и отвечаю на StackOverflow. Иногда занимаюсь ревью постов на StackOverflow.
\end{minipage}

%----------------------------------------------------------------------------------------

\end{document}
