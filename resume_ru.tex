%%%%%%%%%%%%%%%%%%%%%%%%%%%%%%%%%%%%%%%%%
% Developer CV
% LaTeX Template
% Version 1.0 (28/1/19)
%
% This template originates from:
% http://www.LaTeXTemplates.com
%
% Authors:
% Jan Vorisek (jan@vorisek.me)
% Based on a template by Jan Küster (info@jankuester.com)
% Modified for LaTeX Templates by Vel (vel@LaTeXTemplates.com)
%
% License:
% The MIT License (see included LICENSE file)
%
%%%%%%%%%%%%%%%%%%%%%%%%%%%%%%%%%%%%%%%%%

%----------------------------------------------------------------------------------------
%	PACKAGES AND OTHER DOCUMENT CONFIGURATIONS
%----------------------------------------------------------------------------------------

\documentclass[10pt]{tpl/developercv} % Default font size, values from 8-12pt are recommended
\usepackage[russian]{babel}
\usepackage[hidelinks,
            colorlinks = true,
            linkcolor = blue,
            urlcolor  = blue,
            citecolor = blue,
            anchorcolor = blue]{hyperref}
\usepackage{datetime}
\usepackage{texdate}
\initcurrdate
\def\setdateformat{d-m-Y}
%----------------------------------------------------------------------------------------


%----------------------------------------------------------------------------------------
% Header & Footer
%----------------------------------------------------------------------------------------
\usepackage{fancyhdr}

\pagestyle{fancy}
\fancyhf{}
\rhead{\href{https://rame0.github.io/resume-latex/resume_ru.pdf}{Скачать актуальную версию}}
\lhead{Док. обновлен: \printdate\ \currenttime}
% \rfoot{Page \thepage}
%----------------------------------------------------------------------------------------

\begin{document}

%----------------------------------------------------------------------------------------
%	TITLE AND CONTACT INFORMATION
%----------------------------------------------------------------------------------------

\begin{minipage}[t]{0.44\textwidth} % 45% of the page width for name
	\vspace{-\baselineskip} % Required for vertically aligning minipages

	% If your name is very short, use just one of the lines below
	% If your name is very long, reduce the font size or make the minipage wider and reduce the others proportionately
	\colorbox{black}{{\Huge\textcolor{white}{\textbf{\MakeUppercase{Рамиль}}}}} % First name

	\colorbox{black}{{\Huge\textcolor{white}{\textbf{\MakeUppercase{Алиякберов}}}}} % Last name

	\vspace{6pt}

	{\huge Full stack разработчик} % Career or current job title
\end{minipage}
\begin{minipage}[t]{0.275\textwidth} % 27.5% of the page width for the first row of icons
	\vspace{-\baselineskip} % Required for vertically aligning minipages

	% The first parameter is the FontAwesome icon name, the second is the box size and the third is the text
	% Other icons can be found by referring to fontawesome.pdf (supplied with the template) and using the word after \fa in the command for the icon you want
	\icon{MapMarker}{12}{\ Россия, Москва}

	\href{mailto:r@me0.biz}{\icon{At}{12}\ r@me0.biz}

	\href{tel:+79253943150}{\icon{Phone}{12}\ +7 925 394-31-50}
\end{minipage}
\begin{minipage}[t]{0.275\textwidth} % 27.5% of the page width for the second row of icons
	\vspace{-\baselineskip} % Required for vertically aligning minipages

	% The first parameter is the FontAwesome icon name, the second is the box size and the third is the text
	% Other icons can be found by referring to fontawesome.pdf (supplied with the template) and using the word after \fa in the command for the icon you want
	\href{https://www.rame0.ru/}{\icon{Globe}{12}\ www.rame0.ru}

	\href{https://github.com/rame0}{\icon{Github}{12}\ github.com/rame0}

	\href{https://stackoverflow.com/users/513723/rame0}{\icon{StackOverflow}{12}\ StackOverflow}
\end{minipage}

\vspace{0.5cm}

%----------------------------------------------------------------------------------------
%	INTRODUCTION, SKILLS AND TECHNOLOGIES
%----------------------------------------------------------------------------------------

\cvsect{Кто я?}


\begin{minipage}[t]{0.45\textwidth} % 40% of the page width for the introduction text
	\vspace{-\baselineskip} % Required for vertically aligning minipages

	Разработчик с стажем более 10 лет.
  \bigskip

  На последних курсах университета работал C\# разработчиком.

  После службы в ВС РФ работал "'отделом разработки"' в небольшой семейной компании (PHP, JS, CSS).

  После этого некоторое время занимался SEO (продвижение сайтов в поисковых системах) и преподавал его, совмещая с разработкой сайтов на фрилансе.

  На данный момент работаю аутсорс-разработчиком в оптовой компании (PHP, JS, SCSS, ...).

	Основные используемые технологии: {PHP 7.4,} {JS (в основном jQuery)}, SCSS, Docker, NPM, Composer, Gulp, {Git (GitHub)}.
	Периодически используемые: Python, редко C\#.
\end{minipage}
\hfill % Whitespace between
\begin{minipage}[t]{0.5\textwidth} % 50% of the page for the skills bar chart
	\vspace{-\baselineskip} % Required for vertically aligning minipages
	\begin{barchart}{5.5}
		\baritem{JavaScript}{90}
		\baritem{PHP}{100}
		\baritem{MYSQL}{100}
		\baritem{SASS/LESS}{70}
		\baritem{Git}{60}
		\baritem{Docker(compose)}{50}
		\baritem{NPM}{80}
		\baritem{Composer}{80}
	\end{barchart}
\end{minipage}

\begin{center}
	\bubbles{5/PHPStorm, 4/Linux DE, 3/Linux Server, 5/Windows}
\end{center}

%----------------------------------------------------------------------------------------
%	EXPERIENCE
%----------------------------------------------------------------------------------------

\cvsect{Опыт}

\begin{entrylist}
	\entry
		{2007 -- 2008

		\footnotesize{part time}}
		{Junior C\# Developer}
		{ООО 'Группа Алланд'}
		{Разработка пользовательских интерфейсов, архитектуры системы и архитектуры баз данных(в меньшей степени). Мы разрабатывали информационные системы, на подобии 1С, но со специфическими задачами. Так же нами разрабатывалась система контроля полного цикла проведения всероссийских интеллектуальных конкурсов для школьников.

		\texttt{C\#}\slashsep\texttt{SQL}\slashsep\texttt{Web/Win-Services}\slashsep\texttt{OLAP}\slashsep\texttt{TFS}}
	\entry
		{2008-2009}
		{Срочная служба}
		{ВС РФ}
		{Ни чего особенного, иногда помогал с презентациями, офисными программами и программами учета.}
	\entry
		{2009 -- 2015}
		{Full stack разработчик}
		{www.evrosite.ru}
		{Работа в небольшой семейной компании. Был единственным разработчиком, помимо изредка привлекаемых фрилансеров.

		Выполнял весь цикл разработки сайтов, начиная с верстки и заканчивая размещением на хостинге/сервере (включая поддержку серверов). В основном, разработка велась на внутренней CMS разработанной предыдущим программистом. Так же, ипользовались такие CMS, как: OpenCart, 1C Битрикс, AmiroCMS, Wordpress. В сложные времена, так же, выполнял работу продажника, менеджера проектов и другие обязанности.

		\texttt{PHP}\slashsep\texttt{JS}\slashsep\texttt{CSS}\slashsep\texttt{MySQL}\slashsep\texttt{Linux}}

	\entry
		{4/2013 -- 2/2018

		\footnotesize{part time}}
		{SEO специалист/Full stack}
		{Фриланс}
		{С конца 2013 года, когда кол-во заказов стало уменьшаться, заинтересовался SEO продвижением и, понемногу, начала оказывать услуги SEO.\\
		С 2015 основную часть заказов составляли заказы по SEO. С уклоном в фишки, которые можно внедрять с позиции разработчика. Так сказать, базовое продвижение, о котором большинство разработчиков сайтов даже не задумывалось.\\
		По прежнему занимался разработкой сайтов но уже в меньшем количестве (до 10 сайтов в год). В основном делала сайты на AmiroCMS и, иногда, на 1С Битрикс

		\texttt{PHP}\slashsep\texttt{JS}\slashsep\texttt{CSS}\slashsep\texttt{MySQL}\slashsep\texttt{Linux}}
	\entry
		{3/2015 -- 4/2017

		\footnotesize{part time}}
		{Преподаватель SEO}
		{ООО "'ТопЭксперт"' (www.topexpert.pro)}
		{Переподавал базовые знания по Поисковой оптимизации -- SEO (Search Engine Optimisation) -- на образовательной платформе TopЭксперт. Провел в общей сложности 7 потоков, более 100 человек. В обязанности входило проведение уроков теория/практика и проверка домашних заданий. Основной упор в базовых знаниях делался на понимании что такое поисковая система и как она работает, а так же на том, какие ошибки (с точки зрения SEO) допускают разработчики сайтов и как их исправить. Принимал участие в разработке плана обучения, самого курса, а так же доп. материалов(презентаций, примеров и прочего). Написал руководство по составлению семмантического ядра сайта.}
	\entry
		{3/2016 -- 2/2018

		\footnotesize{part time}}
		{Full stack outsource}
		{ООО "'Интерьер комфорт"' }
		{Помощь в поддержке серверов. Доработка и поддержка 2 интернет-магазинов. Поддерка системы управления складом и оптовых продаж.

		\texttt{PHP}\slashsep\texttt{JS}\slashsep\texttt{SCSS}\slashsep\texttt{MySQL}\slashsep\texttt{Linux}}
	\entry
		{2/2018 -- present}
		{Full stack outsource}
		{ООО "'Интерьер комфорт"' }
		{Разработка с нуля системы складов и оптовых продаж. Разработка внутренней CMS для данной системы.

		Перевод интернет магазина на внутреннюю CMS.

		Выделение ядна внутренней CMS в отдельный проект.

		Разработка системы для управления перевозками (по сути транспортной компании) на базе разработанного ядра.

		С начала 2021 года 1 программист в подчинении.

		\texttt{PHP}\slashsep\texttt{JS}\slashsep\texttt{SCSS}\slashsep\texttt{MySQL}\slashsep\texttt{Linux}}
\end{entrylist}

%----------------------------------------------------------------------------------------
%	EDUCATION
%----------------------------------------------------------------------------------------

\cvsect{Образование}

\begin{entrylist}
	\entry
		{2003 -- 2008}
		{Мат. обеспечение и админ. ИС}
		{УлГУ}
		{Получил диплом о высшем образовании по специальности Математическое обеспечение и администрирование информационных систем в Ульяновском Государственном Университете.

		Тема диплома: Разработка автоматизированной системы технической поддержки пользователей, на основе встраиваемого интернет пейджера (.NET).

		Технологии: C\#, MSSQL, Client-Server
		}
	\entry
		{constant}
		{Самообразование}
		{}
		{Стараюсь изучать новые технологии и улучшать знания в имеющихся. К сожалению, не всегда получается изучить то, что на данный момент не применимо. Например, давно хочется изучить ReactJS, но дальше чем пройти пару туториалов дело не заходило, т.к. пока это не применимо к рабочим проектам.}
\end{entrylist}

%----------------------------------------------------------------------------------------
%	ADDITIONAL INFORMATION
%----------------------------------------------------------------------------------------

\begin{minipage}[t]{0.3\textwidth}
	\vspace{-\baselineskip} % Required for vertically aligning minipages

	\cvsect{Languages}

	\textbf{Русский} - родной\\
	\textbf{English} - говорю/пишу\\
\end{minipage}
\hfill
\begin{minipage}[t]{0.3\textwidth}
	\vspace{-\baselineskip} % Required for vertically aligning minipages

	\cvsect{Хобби}

	Часто смотрю что-то образовательное. Изучаю разные штуки, в основном связанные с компьютерами. Например Pen-тестинг или 3d моделирование.
	Реже - читаю sci-fi и фентези.
\end{minipage}
\hfill
\begin{minipage}[t]{0.3\textwidth}
	\vspace{-\baselineskip} % Required for vertically aligning minipages

	\cvsect{Non profit}

	Изредки пишу что то в блог и отвечаю на StackOverflow. Иногда занимаюсь ревью постов на StackOverflow.
\end{minipage}

%----------------------------------------------------------------------------------------

\end{document}
